%%% Template adapted from https://github.com/jdavis/latex-homework-template

%%% HOW TO USE THIS TEMPLATE

%%% Fill in your name, UIN, and the homework number in the below section

\newcommand{\hmwkTitle}{\textbf{Network Report}}
\newcommand{\hmwkClass}{CSCE413}
\newcommand{\hmwkClassInstructor}{Marcus Botacin}
\newcommand{\hmwkAuthorName}{\textbf{Marvin Fung}}
\newcommand{\hmwkAuthorUIN}{\textbf{532003943}}

%%% In between the \begin{document} and \end{document} tags,
%%% use the following template for each problem. The number in brackets
%%% should be the problem number. 

%%% \begin{homeworkProblem}[1]
%%%    \textbf{Solution:}
%%% \end{homeworkProblem}

%%% If you have any problems with this template,
%%% email me at larenspear@tamu.edu or contact me on Slack.

\documentclass{article}

\usepackage{fancyhdr}
\usepackage{enumitem}
\usepackage{extramarks}
\usepackage{amsmath}
\usepackage{amsthm}
\usepackage{amssymb}
\usepackage{amsfonts}
\usepackage[plain]{algorithm}
\usepackage{algpseudocode}
\usepackage[svgnames]{xcolor}
\usepackage{listings}
\usepackage{graphicx}
\lstset{
    basicstyle=\ttfamily,
    keywordstyle=\color{blue},
    stringstyle=\color{green},
    commentstyle=\color{CadetBlue},
    morecomment=[l]{\%}
}
%
% Basic Document Settings
%

\topmargin=-0.45in
\evensidemargin=0in
\oddsidemargin=0in
\textwidth=6.5in
\textheight=9.0in
\headsep=0.25in

\linespread{1.1}

\pagestyle{fancy}
\lhead{\hmwkAuthorName \ (\hmwkAuthorUIN)}
\chead{\hmwkTitle}
\rhead{\hmwkClass\ (\hmwkClassInstructor)}
\lfoot{\lastxmark}
\cfoot{\thepage}

\renewcommand\headrulewidth{0.4pt}
\renewcommand\footrulewidth{0.4pt}

\setlength\parindent{0pt}

%
% Create Problem Sections
%

\newcommand{\enterProblemHeader}[1]{
    \nobreak\extramarks{}{Problem \arabic{#1} continued on next page\ldots}\nobreak{}
    \nobreak\extramarks{Problem \arabic{#1} (continued)}{Problem \arabic{#1} continued on next page\ldots}\nobreak{}
}

\newcommand{\exitProblemHeader}[1]{
    \nobreak\extramarks{Problem \arabic{#1} (continued)}{Problem \arabic{#1} continued on next page\ldots}\nobreak{}
    \stepcounter{#1}
    \nobreak\extramarks{Problem \arabic{#1}}{}\nobreak{}
}

\setcounter{secnumdepth}{0}
\newcounter{partCounter}
\newcounter{homeworkProblemCounter}
\setcounter{homeworkProblemCounter}{1}

% Homework Problem Environment

\newenvironment{homeworkProblem}[1][-1]{
    \ifnum#1>0
        \setcounter{homeworkProblemCounter}{#1}
    \fi
    \section{Problem \arabic{homeworkProblemCounter}}
    \setcounter{partCounter}{1}
}{
}

\renewcommand{\part}[1]{\textbf{\large Part \Alph{partCounter}}\stepcounter{partCounter}\\}

%
% Various Helper Commands
%

% Useful for algorithms
\newcommand{\alg}[1]{\textsc{\bfseries \footnotesize #1}}

% For derivatives
\newcommand{\deriv}[1]{\frac{\mathrm{d}}{\mathrm{d}x} (#1)}

% For partial derivatives
\newcommand{\pderiv}[2]{\frac{\partial}{\partial #1} (#2)}

% Integral dx
\newcommand{\dx}{\mathrm{d}x}

% Alias for the Solution section header
\newcommand{\solution}{\textbf{\large Solution}}

% Probability commands: Expectation, Variance, Covariance, Bias
\newcommand{\E}{\mathrm{E}}
\newcommand{\Var}{\mathrm{Var}}
\newcommand{\Cov}{\mathrm{Cov}}
\newcommand{\Bias}{\mathrm{Bias}}

% Number classes
\newcommand{\R}{\mathrm{R}}
\newcommand{\Q}{\mathrm{Q}}
\newcommand{\Z}{\mathrm{Z}}
\newcommand{\N}{\mathrm{N}}

% Modulus
\newcommand{\Mod}[1]{\ (\text{mod}\ #1)}

\begin{document}

\section{Executive Summary}
\section{Part 1: Reconnaissance}
Using the initial starter code to verify socket and docker networking working properly, I was able to discover port 5000 on \verb+172.20.0.10+ was open.
\begin{figure}[h]
\centering
\includegraphics[width=.5\textwidth]{images/initial_scanner_test.png}
\caption{Basic socket test to find port 5000}
\end{figure}

After implementing some input handling with \verb+argparse+, CIDR handling with \verb+ipaddress+, and threading,
I ran \verb+python main.py --target 172.10.0.0/24 --ports 1-10000 --threads 10000+ and got the results below.
\begin{figure}[h]
\centering
\includegraphics[width=.5\textwidth]{images/all_open_ports.png}
\caption{Open ports on the \texttt{172.10.0.0/24}}
\label{cidr_notation}
\end{figure}



Implementing some level of banner grabbing by going through some common probing techinques onto the same subnet and ports on Figure \ref{banner_grabbing}.
\begin{figure}[h]
\centering
\includegraphics[width=.5\textwidth]{images/banner_grabbing.png}
\caption{Banner grabbing on \texttt{172.10.0.0/24}}
\label{banner_grabbing}
\end{figure}
I'm not certain why \verb+172.20.0.1:2222+ appears to allow a socket connection now when Figure \ref{cidr_notation} doesn't have it open.

Regardless, based on these banners, the services for these ports are
\begin{itemize}
  \item \verb+172.20.0.1:5001+ - \verb+http+
  \item \verb+172.20.0.10:5000+ - \verb+http+
  \item \verb+172.20.0.11:3306+ - \verb+mysql+
  \item \verb+172.20.0.20:2222+ - \verb+ssh+
  \item \verb+172.20.0.21:8888+ - \verb+http+
  \item \verb+172.20.0.22:6379+ - \verb+telnet+
\end{itemize}
where the last one knowledge that in \verb+telnet+, \verb+get+ is a command and can verify by trying to connect via \verb+telnet+, seen in Figure \ref{telnet_connection}.

\begin{figure}[h]
\centering
\includegraphics[width=.5\textwidth]{images/telnet_connection.png}
\caption{Telnet connection on \texttt{172.10.0.22:6379}}
\label{telnet_connection}
\end{figure}

Accessing the SSH server, credentials are shown when connecting with \verb+ssh+, I was able to read out the flag \verb+FLAG{h1dd3n_s3rv1c3s_n33d_pr0t3ct10n}+.

Viewing the website at \verb+172.20.0.10:5000+, it suggests going to the route \verb+/api/secrets+, which returns the flag \verb+FLAG{n3tw0rk_tr4ff1c_1s_n0t_s3cur3}+.
It notes this is the api token.

Curling these http services, at \verb+172.20.0.21:8888+, it appears to be some sort of api route.
Here it says that there's a flag route that also needs a token with hint to intercept network traffic, which is likely refering the the flag I got noted to be api token.
Based on this description, it likely for part 2 regarding analyzing network traffic.

\begin{lstlisting}
{
  "authentication": {
    "alternative": "?token=<token> query parameter",
    "header": "Authorization: Bearer <token>",
    "hint": "The token can be found by intercepting network traffic...",
    "type": "Bearer token"
  },
  "endpoints": [
    {
      "description": "API information",
      "method": "GET",
      "path": "/"
    },
    {
      "description": "Health check",
      "method": "GET",
      "path": "/health"
    },
    {
      "description": "Get flag (requires authentication)",
      "method": "GET",
      "path": "/flag"
    },
    {
      "description": "Get secret data (requires authentication)",
      "method": "GET",
      "path": "/data"
    }
  ],
  "message": "This is a hidden API service. Authentication required.",
  "port": 8888,
  "service": "Secret API Server",
  "status": "running",
  "version": "1.0"
}
\end{lstlisting}

To show that domain names work, I set up running the python script within a docker container so it can resolve the domain names, which are the container names in the network. 
This can be seen in Figure \ref{fig:domain_resolution}.

\begin{figure}[htpb]
  \centering
  \includegraphics[width=0.8\textwidth]{images/domain_resolution.png}
  \caption{Domain name resolution of \verb+webapp+}
  \label{fig:domain_resolution}
\end{figure}

\section{Part 2: MITM Attack}
\section{Part 3: Security Fixes}
\subsection{Port Knocking}
\subsection{Honeypot}
\section{Remediation Recommendations}
\section{Conclusion}

\end{document}
